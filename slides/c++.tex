\documentclass[handout,compress,dvipsnames,pdflatex,beamer]{beamer}
%\documentclass[letterpaper,11pt,notitlepage]{article}\usepackage{beamerarticle}
%\usepackage{amsmath}

\usepackage{listings}
\usepackage{url}

\usepackage{color}
\definecolor{myDarkBlue}{rgb}{0.1,0.1,0.4}
\definecolor{myDarkGrey}{rgb}{0.15,0.15,0.15}
\definecolor{myOrange}{rgb}{0.8,0.5,0.0}
\hypersetup{                  		% beamer colors taken from elsewhere
  hyperindex,%				% works with the beetle colour scheme
  colorlinks,%
  linktocpage,%
  plainpages=true,%
  linkcolor=myOrange,%
  citecolor=myDarkGrey,%
  urlcolor=myDarkBlue,%
  pdfstartview=Fit,%
  pdfview={XYZ null null null}%
}


%\newcommand\code{\bgroup\@codex}
%\def\@codex#1{{\normalfont\ttfamily\hyphenchar\font=-1 #1}\egroup}
%\mode<article>{\usepackage[text={6.2in,9in},centering]{geometry}}
%\mode<beamer>{\usetheme{Boadilla}\usecolortheme{seahorse}\usecolortheme{rose}}
%\mode<beamer>{\usetheme[secheader]{Boadilla}\usecolortheme{whale}}
%\mode<beamer>{\usetheme[secheader]{Madrid}\usecolortheme{whale}}

%\mode<beamer>{\setkeys{Gin}{width=\textwidth}}
%\mode<article>{\setkeys{Gin}{width=0.8\textwidth}}

\mode<presentation>{\usetheme{Warsaw}}


\title[C++ for R Programmers]{C++ for R Programmers}


\author[Dirk Eddelbuettel]{Dr.~Dirk Eddelbuettel\\ \scriptsize\url{edd@debian.org}\\ \url{dirk.eddelbuettel@R-Project.org}}

\date[useR! 2012 @ Vanderbilt]{
  { \small 
    Invited Session: 
    \textsl{What other languages should R users know about?}}\\[20pt] 
  \textsl{useR!} 2012\\ Vanderbilt University \\
  June 14, 2012  }

\begin{document}

%\mode<article>{\maketitle\tableofcontents}

\mode<presentation>{\frame{\titlepage}}

%\mode<presentation>{\frame{\frametitle{Outline}\tableofcontents[pausesections,hideallsubsections]}}

\section[C++?]{Why C++?}
\subsection{Overview}
\begin{frame}
  \frametitle{So ``Why C++'' ?}
  
  \begin{itemize}
  \item Asking Google leads to 51,600,000 hits.
  \item I did not read all of them.
  %\item Some lead to people at most one degree of
  %  separation away from the audience like John D.~Cook.
  \item \href{http://en.wikipedia.org/wiki/C\%2B\%2B}{Wikipedia} starts with
    \emph{C++ (pronounced "cee plus plus") is a statically typed,
      free-form, multi-paradigm, compiled, general-purpose, powerful
      programming language.}  
  \item We could spend this session discussing just that sentence.
  \item C++ is industrial-strength, widely-used, vendor-independent and
    \emph{still evolving}.
  \item In science and research, it is one of the most widely-used
    languages.  If there is something you want to use or connect to, it
    probably has a C/C++ API.
  \end{itemize}
\end{frame}

\subsection{Federation}
\begin{frame}
  \frametitle{Another popular view on ``What C++ is''}
  \framesubtitle{From Scott Meyers highly-regarded``Effective C++''}

  Item 1:   \emph{``View C++ as a federation of languages''}

  \begin{enumerate}
  \item \emph{C} provides a rich inheritance and interoperability as Unix, Windows,
    ... are all build on C
  \item \emph{Object-Oriented C++} just to provide endless discussions about the
    right notion of what OO is or should be (and R really helps here having
    three different ones to offer :-/ )
  \item \emph{Templated C++} which is mighty powerful; template meta
    programming unequalled in other languages
  \item \emph{The STL} which is a specific template library which is powerful but
    has its own conventions
  \end{enumerate}
  
\end{frame}

\section[With R]{Why C++ with R?}
\subsection{Mixing}
\begin{frame}
  \frametitle{Barbells not bullets}
  \begin{itemize}
  \item On my first job, ``barbell'' (portfolios) meant those comprised of
    both long and short duration bonds -- as opposed to ``bullet'' portfolio concentrated at one (middle) duration.
  \item I feel language choice is similar:  It is rare to have one single
    solution for all problems.
  \item Python may be close.  Julia may well get there too.
  \item But I am a realist, and I have \emph{never} been on a project or
    team that was single-language, single-solution.
  \item In practive, people will always mix.  John Chambers has made that
    point eloquently too.
  \item So face this head-on and pick tools \emph{which mix well}.  It so
    happens that I think R and C++ mix well via Rcpp / RInside.
  \end{itemize}
\end{frame}

\subsection{Interchange}
\begin{frame}
  \frametitle{Sending R objects back and forth}
  \framesubtitle{Possible with R's API, easier with Rcpp}
  \begin{itemize}
  \item Essentially, any R object is represented internally as a SEXP.
  \item The \texttt{.Call} interface lets you send SEXPs back and forth.
  \item SEXP can be nested just like R objects: lists of lists of ...
  \item Rcpp makes the interchange of R objects a little easier than the
    plain C API for R. 
  \item ``Empirically speaking'', 68 CRAN packages (as
    of 3 June 2012) using Rcpp seem to agree.
  \item This makes Rcpp the most widely used foreign-language interface
    package for R (as it overtook rJava recently). Of course, there is the
    plain C API...
  \end{itemize}
\end{frame}

\end{document}

%%% Local Variables: 
%%% mode: latex
%%% TeX-master: t
%%% End: 
